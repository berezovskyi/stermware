\title{STermWare: brief semantics description}
\author{
        Ruslan Shevchenko \\
                ruslan@shevchenko.kiev.ua\\
        Kiev, Ukraine
}
\date{\today}



\documentclass[12pt]{article}
\usepackage{listings}

% "define" Scala
\lstdefinelanguage{scala}{
  morekeywords={abstract,case,catch,class,def,%
    do,else,extends,false,final,finally,%
    for,if,implicit,import,match,mixin,%
    new,null,object,override,package,%
    private,protected,requires,return,sealed,%
    super,this,throw,trait,true,try,%
    type,val,var,while,with,yield},
  otherkeywords={=>,<-,<\%,<:,>:,\#,@},
  sensitive=true,
  morecomment=[l]{//},
  morecomment=[n]{/*}{*/},
  morestring=[b]",
  morestring=[b]',
  morestring=[b]"""
}



\begin{document}
\maketitle

\begin{abstract}
 Termware is a term rewriting system which implemented as internal and external DSL.
\end{abstract}

\section{Motivation}

  Rule-based deduction and rewriting is a well known formal method techniques.


  
  


\section{Basic definitions}

   Let's build term algebra on top of some set of constants, whuch constists from
\begin{itemize}
 \item original constants ${c_i} \in C$ Also we
 will need to distinguish some special subset of constants: atoms ${a_i}$
 \item functional term $f_i \in F$  i.e. if $f \in F, c_i \in C (i \in 1..n)$, then $f(c_{1} \dots c_{n}) \in T$
 \item variable terms $x_{i}$ which can be inside variable-bound terms.
 \item variable bounds term, i.e. so-called 'with-expressions': $with(x_1,\dots x_n):t$
 \item let terms, i.e. so-called 'let expressions' $let \{ x_i \leftarrow t_i \}_{1\dots n} t  \in T$
 \item rule terms, i. e. so-called 'rule expressions': 
       $rule(p_1 ; p_2 ; \dots p_n \to r_n : p_{fail})$
 \item application $a b$
\end{itemize}

  Let's define usual term operations:
  \begin{itemize}
    \item s contains t: $s \cap t$
  \end{itemize}

  Transformations:

\begin{itemize}
     \item application reduction
           $$ rule(with(x_i):p_i \to r_i, ... ) q = 
                    \left\{\begin{array}{l l}
                             \sigma r_i & if \;  \sigma=unify(p_i,q)/x_{i} \\
                             p_{fail}   & otherwise \\
                           \end{array}
                    \right\} 
           $$
     \item with propagation
           $$ with(x):rule(p_1 \dots p_{n}, p_{fail}) =
               rule(with(x,p_1) \dots with(x,p_{n}), with(p_{fail})) =
           $$
     \item with chaining
           $$ with(x_1 \dots x_n):with(y_1 \dots y_n):p =
                with(x_1 \dots x_n, y_1 \dots y_n):p $$
     \item with variable elimination
           $$ with(x_{1} \dots x_{i} \dots x_{n}):p = 
              with(x_{1} \dots x_{i-1}, x_{i+1}, \dots x_{n})p \; if x_i \not\in p $$  
\end{itemize}


\section{Related work}

  Pure pattern calculus \cite{Jay05purepattern} is a most close formal system.  It's congruent by 
 price of not 

\bibliographystyle{abbrv}
\bibliography{stermware}

\end{document}
